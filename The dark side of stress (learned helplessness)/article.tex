\documentclass{article}

\usepackage[margin=1.5in]{geometry}
\usepackage[parfill]{parskip}
\usepackage[utf8]{inputenc}


% \usepackage[backend=biber, style=numeric, sorting=none]{biblatex} % Or your preferred style
\usepackage[backend=biber, style=alphabetic, sorting=none]{biblatex}
\addbibresource{references.bib} % Add this line

\usepackage{hyperref}
% \hypersetup{ % Optional: customize link appearance
%     colorlinks=true,
%     linkcolor=blue,   % color of internal links (e.g., toc, sections)
%     citecolor=green,  % color of links to bibliography
%     urlcolor=magenta  % color of external links

\title{The dark side of stress (learned helplessness) }
\author{Ray Peat}
\date{2016}



\begin{document}

\maketitle

% \begin{abstract}
%     Acetylcholine is the "neurotransmitter" of cholinergic nerves, including the parasympathetic system.
%     Cholinesterase (or acetylcholinesterase) is an enzyme that destroys acetylcholine, limiting the action of the cholinergic nerves.
%     Attaching a phosphate group to the cholinesterase enzyme inactivates it, prolonging and intensifying the action of cholinergic stimulation.
% \end{abstract}

\section{Introduction}
\subsection{Paragraph 1}
Acetylcholine is the "neurotransmitter" of cholinergic nerves, including the parasympathetic system.
Cholinesterase (or acetylcholinesterase) is an enzyme that destroys acetylcholine, limiting the action of the cholinergic nerves.
Attaching a phosphate group to the cholinesterase enzyme inactivates it, prolonging and intensifying the action of cholinergic stimulation.

\subsection{Paragraph 2}
The autonomic nervous system has traditionally been divided into the sympathetic-adrenergic system, and the parasympathetic-cholinergic system, with approximately opposing functions, intensifying energy expenditure and limiting energy expenditure, respectively. The hormonal system and the behavioral system interact with these systems, and each is capable of disrupting the others. Disruptive factors in the environment have increased in recent decades.

\section{Body}

\subsection{Environmental Enrichment and Brain Development}
Living is development; the choices we make create our individuality. If genetically identical mice grow up in a large and varied environment, small differences in their experience will affect cell growth in their brains, leading to large differences in their exploratory behavior as they age \cite{Freund2013}. Geneticists used to say that "genes determine our limits," but this experiment shows that an environment can provide both limitations and opportunities for expanding the inherited potential. If our environment restricts our choices, our becoming human is thwarted, the way rats' potentials weren't discovered when they were kept in the standard little laboratory boxes. An opportunity to be complexly involved in a complex environment lets us become more of what we are, more humanly differentiated.

A series of experiments that started at the University of California in 1960 found that rats that lived in larger spaces with various things to explore were better at learning and solving problems than rats that were raised in the standard little laboratory cages \cite{Krech1960}. Studying their brains, they found that the enzyme cholinesterase, which destroys the neurotransmitter, acetylcholine, was increased. They later found that the offspring of these rats were better learners than their parents, and their brains contained more cholinesterase. Their brains were also larger, with a considerable thickening of the cortex, which is considered to be the part mainly responsible for complex behavior, learning and intelligence.

These processes aren't limited to childhood. For example, London taxi drivers who learn all the streets in the city develop a larger hippocampus, an area of the brain involved with memory.

\subsection{Learned Helplessness: Historical Context and Cholinergic Link}
The 1960s research into environmental enrichment coincided with political changes in the US, but it went against the dominant scientific ideas of the time. Starting in 1945, the US government had begun a series of projects to develop techniques of behavior modification or mind control, using drugs, isolation, deprivation, and torture. In the 1950s, psychiatry often used lobotomies (about 80,000, before they were generally discontinued in the 1980s) and electroconvulsive "therapy," and university psychologists tortured animals, often as part of developing techniques for controlling behavior.

The CIA officially phased out their MKultra program in 1967, but that was the year that Martin Seligman, at the University of Pennsylvania, popularized the idea of "learned helplessness." He found that when an animal was unable to escape from torture, even for a very short time, it would often fail to even try to escape the next time it was tortured. Seligman's lectures have been attended by psychologists who worked at Guantanamo, and he recently received a no-bid Pentagon grant of \$31,000,000, to develop a program of "comprehensive soldier fitness," to train marines to avoid learned helplessness.

Curt Richter already in 1957 had described the "hopelessness" phenomenon in rats (``a reaction of hopelessness is shown by some wild rats very soon after being grasped in the hand and prevented from moving. They seem literally to give up,'') and even how to cure their hopelessness, by allowing them to have an experience of escaping once \cite{Richter1957, Richter1958}. Rats which would normally be able to keep swimming in a tank for two or three days, would often give up and drown in just a few minutes, after having an experience of "inescapable stress." Richter made the important discovery that the hearts of the hopeless rats slowed down before they died, remaining relaxed and filled with blood, revealing the dominant activity of the vagal nerve, secreting acetylcholine.

The sympathetic nervous system (secreting noradrenaline) accelerates the heart, and is usually activated in stress, in the "fight or flight" reaction, but this radically different (parasympathetic) nervous activity hadn't previously been seen to occur in stressful situations. The parasympathetic, cholinergic, nervous system had been thought of as inactive during stress, and activated to regulate processes of digestion, sleep, and repair. Besides the cholinergic nerves of the parasympathetic system, many nerves of the central nervous system also secrete acetylcholine, which activates smooth muscles, skeletal muscles, glands, and other nerves, and also has some inhibitory effects. The parasympathetic nerves also secrete the enzyme, cholinesterase, which destroys acetylcholine. However, many other types of cell (red blood cells, fibroblasts, sympathetic nerves, marrow cells), maybe all cells, can secrete cholinesterase.

Because cholinergic nerves have been opposed to the sympathetic, adrenergic, nerves, there has been a tendency to neglect their nerve exciting roles, when looking at causes of excitotoxicity, or the stress-induced loss of brain cells. Excessive cholinergic stimulation, however, can contribute to excitotoxic cell death, for example when it's combined with high cortisol and/or hypoglycemia.

\subsection{Cholinergic System, Dementia, and Treatment Controversies}
Drugs that block the stimulating effects of acetylcholine (the anticholinergics) as well as chemicals that mimic the effects of acetylcholine, such as the organophosphate insecticides, can impair the ability to think and learn. This suggested to some people that age-related dementia was the result of the deterioration of the cholinergic nerves in the brain. Drugs to increase the stimulating effects of acetylcholine in the brain (by inactivating cholinesterase) were promoted as treatment for Alzheimer's disease.

Although herbal inhibitors were well known, profitable new drugs, starting with Tacrine, were put into use. It was soon evident that Tacrine was causing serious liver damage, but wasn't slowing the rate of mental deterioration.

As the failure of the cholinergic drug Tacrine was becoming commonly known, another drug, amantadine (later, the similar memantine) was proposed for combined treatment. In the 1950s, the anticholinergic drug atropine was proposed a few times for treating dementia, and amantadine, which was also considered anticholinergic, was proposed for some mental conditions, including Creutzfeldt-Jacob Disease \cite{Sanders1973}. It must have seemed odd to propose that an anticholinergic drug be used to treat a condition that was being so profitably treated with a pro-cholinergic drug, but memantine came to be classified as an anti-excitatory "NMDA blocker," to protect the remaining cholinergic nerves, so that both drugs could logically be prescribed simultaneously. The added drug seems to have a small beneficial effect, but there has been no suggestion that this could be the result of its previously-known anticholinergic effects.

Over the years, some people have suspected that Alzheimer's disease might be caused partly by a lack of purpose and stimulation in their life, and have found that meaningful, interesting activity could improve their mental functioning. Because the idea of a "genetically determined hard-wired" brain is no longer taught so dogmatically, there is increasing interest in this therapy for all kinds of brain impairment. The analogy to the Berkeley enrichment experience is clear, so the association of increasing cholinesterase activity with improving brain function should be of interest.

The after-effect of poisoning by nerve gas or insecticide has been compared to the dementia of old age. The anticholinergic drugs are generally recognized for protecting against those toxins. Traumatic brain injury, even with improvement in the short term, often starts a long-term degenerative process, greatly increasing the likelihood of dementia at a later age. A cholinergic excitotoxic process is known to be involved in the traumatic degeneration of nerves \cite{Lyeth1992}, and the use of anticholinergic drugs has been recommended for many years to treat traumatic brain injuries (e.g., \cite{Ward1950}; \cite{Ruge1954}; \cite{Hayes1986}).

\subsection{Enrichment, Deprivation, and Cholinergic Activity}
In 1976 there was an experiment \cite{Rosellini1976} that made an important link between the enrichment experiments and the learned helplessness experiments. The control animals in the enrichment experiments were singly housed, while the others shared a larger enclosure. In the later experiment, it was found that the rats "who were reared in isolation died suddenly when placed in a stressful swimming situation," while the group-housed animals were resistant, effective swimmers. Enrichment and deprivation have very clear biological meaning, and one is the negation of the other.

The increase of cholinesterase, the enzyme that destroys acetylcholine, during enrichment, serves to inactivate cholinergic processes. If deprivation does its harm by increasing the activity of the cholinergic system, we should expect that a cholinergic drug might substitute for inescapable stress, as a cause of learned helplessness, and that an anticholinergic drug could cure learned helplessness. Those tests have been done: "Treatment with the anticholinesterase, physostigmine, successfully mimicked the effects of inescapable shock." "The centrally acting anticholinergic scopolamine hydrobromide antagonized the effects of physostigmine, and when administered prior to escape testing antagonized the disruptive effects of previously administered inescapable shock." \cite{Anisman1981}.

This kind of experiment would suggest that the anticholinesterase drugs still being used for Alzheimer's disease treatment aren't biologically helpful. In an earlier newsletter I discussed the changes of growth hormone, and its antagonist somatostatin, in association with dementia: Growth hormone increases, somatostatin decreases. The cholinergic nerves are a major factor in shifting those hormones in the direction of dementia, and the anticholinergic drugs tend to increase the ratio of somatostatin to growth hormone. Somatostatin and cholinesterase have been found to co-exist in single nerve cells \cite{Delfs1984}.

\subsection{Hormonal Influences: Estrogen, Progesterone, and DHEA}
Estrogen, which was promoted so intensively as prevention or treatment for Alzheimer's disease, was finally shown to contribute to its development. One of the characteristic effects of estrogen is to increase the level of growth hormone in the blood. This is just one of many ways that estrogen is associated with cholinergic activation. During pregnancy, it's important for the uterus not to contract. Cholinergic stimulation causes it to contract; too much estrogen activates that system, and causes miscarriage if it's excessive. An important function of progesterone is to keep the uterus relaxed during pregnancy. In the uterus, and in many other systems, progesterone increases the activity of cholinesterase, removing the acetylcholine which, under the influence of estrogen, would cause the uterus to contract.

Progesterone is being used to treat brain injuries, very successfully. It protects against inflammation, and in an early study, compared to placebo, lowered mortality by more than half. It's instructive to consider its anticholinergic role in the uterus, in relation to its brain protective effects. When the brain is poisoned by an organophosphate insecticide, which lowers the activity of cholinesterase, seizures are likely to occur, and treatment with progesterone can prevent those seizures, reversing the inhibition of the enzyme (and increasing the activity of cholinesterase in rats that weren't poisoned) \cite{Joshi2010}. Similar effects of progesterone on cholinesterase occur in menstrually cycling women \cite{Fairbrother1989}, implying that this is a general function of progesterone, not just something to protect pregnancy. Estrogen, with similar generality, decreases the activity of cholinesterase. DHEA, like progesterone, increases the activity of cholinesterase, and is brain protective \cite{Aly2011}.

Brain trauma consistently leads to decreased activity of this enzyme \cite{Ostberg2011, Donat2007}, causing the acetylcholine produced in the brain to accumulate, with many interesting consequences. In 1997, a group \cite{Pike1997} created brain injuries in rats to test the idea that a cholinesterase inhibitor would improve their recovery and ability to move through a maze. They found instead that it reduced the cognitive ability of both the injured and normal rats. An anticholinergic drug, selegeline (deprenyl) that is used to treat Parkinson's disease and, informally, as a mood altering antiaging drug, was found by a different group \cite{Zhu2000} to improve cognitive recovery from brain injuries.

\subsection{The Role of Nitric Oxide (NO)}
One of acetylcholine's important functions, in the brain as elsewhere, is the relaxation of blood vessels, and this is done by activating the synthesis of NO, nitric oxide. (Without NO, acetylcholine constricts blood vessels; \cite{Librizzi2000}.) The basic control of blood flow in the brain is the result of the relaxation of the wall of blood vessels in the presence of carbon dioxide, which is produced in proportion to the rate at which oxygen and glucose are being metabolically combined by active cells. In the inability of cells to produce CO2 at a normal rate, nitric oxide synthesis in blood vessels can cause them to dilate. The mechanism of relaxation by NO is very different, however, involving the inhibition of mitochondrial energy production \cite{Barron2001}. Situations that favor the production and retention of a larger amount of carbon dioxide in the tissues are likely to reduce the basic "tone" of the parasympathetic nervous system, as there is less need for additional vasodilation.

Nitric oxide can diffuse away from the blood vessels, affecting the energy metabolism of nerve cells \cite{Steinert2010}. Normally, astrocytes protect nerve cells from nitric oxide \cite{Chen2001}, but that function can be altered, for example by bacterial endotoxin absorbed from the intestine \cite{Sola2002} or by amyloid-beta \cite{Tran2001}, causing them to produce nitric oxide themselves.

Nitric oxide is increasingly seen as an important factor in nerve degeneration \cite{Doherty2011}. Nitric oxide activates processes \cite{Obukuro2013} that can lead to cell death. Inhibiting the production of nitric oxide protects against various kinds of dementia \cite{SharmaSharma2013, SharmaSingh2013}. Brain trauma causes a large increase in nitric oxide formation, and blocking its synthesis improves recovery \cite{Huttemann2008, Gahm2006}. Organophosphates increase nitric oxide formation, and the protective anticholinergic drugs such as atropine reduce it \cite{Chang2001, Kim1997}. Stress, including fear \cite{Campos2013} and isolation \cite{Zlatkovic2013} can activate the formation of nitric oxide, and various mediators of inflammation also activate it. The nitric oxide in a person's exhaled breath can be used to diagnose some diseases, and it probably also reflects the level of their emotional well-being.

\subsection{Protective Mechanisms and Interventions}
The increase of cholinesterase by enriched living serves to protect tissues against an accumulation of acetylcholine. The activation of nitric oxide synthesis by acetylcholine tends to block energy production, and to activate autolytic or catabolic processes, which are probably involved in the development of a thinner cerebral cortex in isolated or stressed animals. Breaking down acetylcholine rapidly, the tissue renewal processes are able to predominate in the enriched animals.

Environmental conditions that are favorable for respiratory energy production are protective against learned helplessness and neurodegeneration, and other biological problems that involve the same mechanisms. Adaptation to high altitude, which stimulates the formation of new mitochondria and increased thyroid (T3) activity, has been used for many years to treat neurological problems, and the effect has been demonstrated in animal experiments \cite{Manukhina2010}. Bright light can reverse the cholinergic effects of inescapable stress \cite{Flemmer1990}.

During the development of learned helplessness, the T3 level in the blood decreases \cite{Helmreich2006}, and removal of the thyroid gland creates the "escape deficit," while supplementing with thyroid hormone before exposing the animal to inescapable shock prevents its development \cite{Levine1990}. After learned helplessness has been created in rats, supplementing with T3 reverses it \cite{Massol1987a, Massol1988}.

Hypothyroidism and excess cholinergic tone have many similarities, including increased formation of nitric oxide, so that similar symptoms, such as muscle inflammation, can be produced by cholinesterase inhibitors such as Tacrine, by increased nitric oxide, or by simple hypothyroidism \cite{Jeyarasasingam2000, Franco2006}.

Insecticide exposure has been suspected to be a factor in the increased incidence of Alzheimer's disease \cite{Zaganas2013}, but it could be contributing to many other problems, involving inflammation, edema, and degeneration. Another important source of organophosphate poisoning is the air used to pressurize airliners, which can be contaminated with organophosphate fumes coming from the engine used to compress it.

\subsection{Conclusion and Recommendations}
Possibly the most toxic component of our environment is the way the society has been designed, to eliminate meaningful choices for most people. In the experiment of Freund, et al. \cite{Freund2013}, some mice became more exploratory because of the choices they made, while others' lives became more routinized and limited. Our culture reinforces routinized living. In the absence of opportunities to vary the way you work and live to accord with new knowledge that you gain, the nutritional, hormonal and physical factors have special importance.

Supplements of thyroid and progesterone are proven to be generally protective against the cholinergic threats, but there are many other factors that can be adjusted according to particular needs. Niacinamide, like progesterone, inhibits the production of nitric oxide, and also like progesterone, it improves recovery from brain injury \cite{Hoane2008}. In genetically altered mice with an Alzheimer's trait, niacinamide corrects the defect \cite{Green2008}. Drugs such as atropine and antihistamines can be used in crisis situations. Bright light, without excess ultraviolet, should be available every day.

The cholinergic system is much more than a part of the nervous system, and is involved in cell metabolism and tissue renewal. Most people can benefit from reducing intake of phosphate, iron, and polyunsaturated fats (which can inhibit cholinesterase; \cite{Willis2009}), and from choosing foods that reduce production and absorption of endotoxin. And, obviously, drugs that are intended to increase the effects of nitric oxide and acetylcholine should be avoided.


\newpage

\printbibliography


\end{document}